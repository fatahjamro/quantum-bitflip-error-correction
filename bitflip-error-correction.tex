\documentclass[a4paper, 12pt]{scrartcl}

\title{Quantum Error Correction}
\subtitle{Quantum Bit-Flip Error Correction}
\author{Abdul Fatah Jamro}
\date{\today}

\begin{document}


\maketitle
  In this article, We consider quantum error correction. 
  Perticularly, We describe bitflip error in quantum computing 
  and correct it using qiskit python.

\section{Introduction}
  Qubit is very fragile in nature. It is valnurable to decoherence and noise.
  Qubit mainly come across with two type of errors bit-flip and phase-flip errors.
  We can correct qubit errors by quantum error correction techinques. 
  Bit-flip error can be corrected using quantum circuit called bit flip code.

  \section{Bit flip error}
  Bit flip error is such an error in which a bit is flipped from one state to 
  its opposite state. For instance 0 flips to 1 state $0  -> 1$
  In quantum circuit, We use 3 qubits, one logical qubit and two ancillary (extra) qubits.
\section{Bit flip circuit}
  We encode the logical qubit by (entangling) applying control NOT $CX or CNOT$ gate with other two ancillary qubits.
  This process is also called encoding. After encoding if main qubit or logical qubit is flipped
  we repeat the process of applying CNOT gates just like mirroring the encoding process.
  This mirroring is also called decoding. At the end a taffoli gate is applied that is control-control-NOT gate $CCX$
  This circuit is called bit flip error correcting circuit. This circuit of 3 qubits has capacity 
  to correct the error if one qubit is flipped only. It can not correct the qubit if
  more than one qubits are flipped.

  Python code of this quantum circuit is written and availbe in jupyter notebook file in this repository.

\end{document}